\section*{Lecture \#7: Lyapunov Direct Method Stability}


\subsection*{Definition of Stability($\epsilon - \delta$ )}
Recall that we can a \underline{stable} system to be one in where

$$
\forall \epsilon > 0 \text{, } \exists \; \delta = \delta(\epsilon) > 0
$$

\noindent Such that for a given

$$
\left\Vert x(0) \right\Vert < \delta
$$

\noindent There exists a $\epsilon$ such that

$$
\left\Vert x(t) \right\Vert < \epsilon
$$

\subsection*{Definition of Asymptoticlly Stable}
 Further, recall that we can classify a system as asymptotically stable when ...

 $$
x(t) \rightarrow 0 \text{, } t \rightarrow \infty
 $$

\noindent This is the condition that the system is \underline{attractive}.



\subsection*{Example: Pendulum with Friction}

$$
E(x) = \frac{f}{l}[a - Cos(x_1)] + \frac{1}{2}x_2^2
$$

\noindent \textbf{Properties}

\begin{enumerate}
  \item $E(0,0) = 0 \quad \quad \text{Function is lower bounded by zero} $
  \item $E(x_1, x_2) \geq 0 $
\end{enumerate}

$$
\frac{dE}{dt} < 0 \quad \forall t \rightarrow \infty \quad x =0 \text{ (Asymptoticlly Stable)}
$$

$$
\frac{dE}{dt} = \frac{\partial E}{\partial x_1} \frac{d x_1}{dt} + \frac{\partial E}{\partial x_2} \frac{d x_2}{dt}
$$

\noindent Where $\dot{x_1}$ and $\dot{x_2}$ are the state equations of the system.

$$
\therefore \frac{dE}{dt} = - \frac{k}{m}d_2^2 \leq 0 \; \forall x \in \R^2
$$

\noindent Since $\frac{dE}{dt} \leq 0$, show that $x =0 $ is stable...

\subsubsection*{Remarks on Pendulum:}

\begin{enumerate}
  \item This energy function is nonincreasing but not necessarily decreasing over time.
  \item From physics, we know that the origin is \textbf{asymptotically stable}.
\end{enumerate}

\subsection*{Lyapunov Direction Method for Assessing Stability}

Given the system...

$$
\dot{x} = f(x) \quad \quad f:D \rightarrow \R^n \text{ at, } f(0) = 0
$$

\noindent is locally Lipschitz (solution exists \& is unique), let's define

$$
V:D \rightarrow \R^{+} \quad \text{where } D \subset \R^n
$$

\noindent The time derivative of the $V$ is given by

$$
\frac{dV}{dt} = \frac{\partial V}{\partial x} \frac{d x}{dt} = \frac{\partial V}{\partial x} \cdot f(x)$$

\noindent \underline{NOTE}: $\frac{dV}{dx} = \left[ \frac{\partial V}{\partial x_1} \cdots \frac{\partial V}{\partial x_n} \right]$

\subsubsection*{Lie Derivative Notation}

The Lie derivative notation for the Lyapunov Function is given by

$$
\dot{V} = \frac{\partial V}{\partial x} f(x) = L_f V
$$


\subsection*{Lyapunov Stability Theorem:}

\newtheorem{theorem}{}

\begin{theorem}
Let $x =0$ be an equilibrium point of $\dot{x} = f(x)$ and let $V$ be a function $V:D \rightarrow \R$, $V \in C_1$ [continuously differentiable] on $D$ (a neighborhood of $X =0$), $V(0) = 0$, $V(x)>0$, $\forall x \in D - \{ 0 \}$
\end{theorem}


\begin{enumerate}
  \item If $\dot{V} = \frac{\partial V}{\partial x}f(x) \leq 0$, $\forall x \in D$, then $x =0$ is \underline{stable}.
  \item If $\dot{V} \leq 0$, $\forall x \in D$, then $x =0$ is \underline{asymptotically stable}.
\end{enumerate}

\subsubsection*{Remarks of Lyapunov's Theorem:}

\begin{itemize}
  \item This theorem does not prove ``necessary condition'', its only provides sufficient condition for proving $x =0 $ is stable or A.S.
  \item No connection between the domain $D$ and the $\epsilon-\delta$ definition of stability.
\end{itemize}

\subsection*{Pendulum with Friction}

Goal: Show that $x =0$ is asymptotically stable.

$$
V(x) = \frac{1}{2} x^TPx + \frac{g}{x}(1 - Cos(x_1))
$$

\noindent Where $P$ is a positive definite matrix

$$
\therefore \quad P =
\begin{bmatrix}
P_{11} & P_{12} \\
P_{21} & P_{22}
\end{bmatrix}  =P^T \succ 0
$$

\noindent Such that $P_{22} = 1$, $P_{11} = \frac{k}{m}\cdot P_{12}$, $P_{12} = P_{21} = .5\frac{k}{m}$, which define the Lyapunov Candidate function. After taking the time derivative of this function, we obtain that

$$
\dot{V}(x) = -.5 \frac{g}{l}\frac{k}{m}x_1 Sin(x_1) - .5 \frac{k}{m}x_2^2
$$

\noindent Which is the function that we want to verify is negative semidefinite(stable) or negative definite (asymptotically stable).

$$
x_1sin(x_1)> 0 \quad \text{ for } 0<|x_1|< \pi
$$

\noindent Such that

$$
D = \{ x \in \R^2 | \left\Vert x_1 \right\Vert < \pi \}
$$


\begin{enumerate}
  \item $V$ is a positive definite function
  \item $\dot{V}$ is a negative definite function over the specified domain. Since $x =0 $ the system is asymptotically stable, via the Lyapunov Theorem.
\end{enumerate}

\subsection*{Important Definitions}

\begin{enumerate}
  \item \underline{Lyapunov function candidate}:
  $V: D \rightarrow \R$, $V \in C_1$ on D, $V(0) = 0$, $V(x)>0$, $\forall x \in D - \{ 0 \}$
  \item \underline{Lyapunov Function}: $\dot{V}\leq 0 $, $\forall x \in D$
  \item \underline{Level Set of $V$}: $\Omega_c := \{ x \in D | V(x) \leq C \}$
\end{enumerate}


\subsection*{Proof of Lyapunov Stability:}

\begin{enumerate}
  \item \textbf{Stability:} \\
  Given $\epsilon > 0$, choose $r \in (0, \epsilon)$ s.t. $B_r := \{ x \in \R^n | \left\Vert x \right\Vert < r\}$, where $B_r \subset D$.
  \item Let $\alpha = min_{\left\Vert x \right\Vert  = r} V(x) \Rightarrow \alpha > 0$
  \item Choos $\beta \in (0, \alpha)$ such that we have level set $\Omega_{\beta}$ such that $\Omega \subset B_r$
  \item Any trajectory starting in $\Omega_{\beta}$ at $t = 0$, staus in $\Omega_{\beta}$ for $t \geq 0$.

  $$
\dot{V}(x) \leq 0 \Rightarrow V(x(t)) \leq V(x(0)) \leq \beta \quad \quad \forall \geq 0
  $$
\end{enumerate}

\noindent $\Omega_{\beta}$ is a compact set (1) Closed \& (2) Bounded. This implies $\dot{x}=f(x)$ has a \underline{unique solution} if $x(0) \in \Omega_{\beta}$. \\

$$
V is continuous \& V(0)=0 \quad \exists \delta > 0
$$

$$
\left\Vert x(0) \right\Vert \leq \delta \Rightarrow V(x) < \beta
$$

$$
B_{\delta} \subset \Omega_{\beta} \subset B_r
$$


$$
\begin{array}{l}
x(0) \in B_{8} \rightarrow x(0) \in \Omega_{\delta} \quad \therefore x(t) \in \Omega_{B} \\
x(t) \in B_{r}, \forall t \geq 0 \\
\|x(0) \mid<\delta \rightarrow\| x(t) \|<r \leqslant \varepsilon \quad \forall t \geq 0 \\
x=0 \text { is stable }
\end{array}
$$
