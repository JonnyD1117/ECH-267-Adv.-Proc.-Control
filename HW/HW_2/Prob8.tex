\section*{Problem 8}


Suppose the set $M$ in LaSalle's theorem consists of a finite number of isolated points. Show that $lim_{t \rightarrow \infty} x(t)$ exists and equals one of these points.\\


\noindent \textbf{Solution:}

In LaSalle's Theorem, the set $M$ is the largest \underline{invariant set} in the set $E$. Where $E$ is the set of all points in the domain $\sigma$ where $\dot{V}(x) = 0$. \\

We know from Lyapunov's Theorem, that $\dot{V}(x(t))$ is negative semi definite function, which means that the Lyapunov candidate function $V$ decreases with time. Due to these properties, and since the solution $x(t)$ starts in the \textbf{invariant set}  $\sigma$, the solution $x(t)$ must \underline{exist} and be bounded inside this set, due to the property that all any trajectory starting at $x(0)$ in the set $\sigma$ must say in $\sigma$.  \\

According to, Lemma 4.1, the solution $x(t)$ will approach its positive limit set as $t\rightarrow \infty$. Since the limit set $L^+$ must comply with the Lyapunov's Theorem such that $\dot{V}(x) \leq 0 $, we know can show that the largest limit set will occur at equality when when the function is not decreasing, $v(x) = 0$. Since we know from Lemma 4.1 that the solution $x(t)$ must approach its limit set $L^+$ as $t \rightarrow \infty$, since it is bounded, and since we defined $M$ to be the largest invariant set of $E$ where $v(x) = 0$ holds, then we can show that the solution $x(t)$ must converge to $M$ as $t \rightarrow \infty$.
