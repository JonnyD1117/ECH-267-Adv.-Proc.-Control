\section*{Problem 2}

Given the scalar system...

$$
\dot{x} = ax^p + g(x)
$$

\noindent Where $p$ is a positive integer such that...

$$
|g(x)| \leq k|x|^{p+1}
$$

\noindent Show that the origin is asymptotically stable if $p$ is odd and $a<0$. Show that it is unstable if $p$ is odd and $a>0$ or $p$ is even and $a\neq0$

\noindent \textbf{Solution:} \\

\noindent Inorder to determine stability, we should begin by establishing a Lyapunov candidate function. Since the function is a scalar equation, we can use...

$$V(x) = \frac{1}{2}\cdot x^2$$

\noindent In order to establish stability using this Lyapunov function, we must differentiate the function with respect to $x$.

$$
\dot{V}(x) = \frac{dV}{dx} \cdot \dot{x}
$$

\noindent Using this definition $\dot{V}$ can be found by differeniating the function, with respect to $x$, and then substituting in the given system.

$$
\frac{dV}{dx} = x
$$

$$
\dot{V} = x[ax^p + g(x)] \leq ax^{p+1} + k|x|^{p+2}
$$

\subsection*{Case 1: ``P'' is odd \& ``a'' less than 0 }

Given the inequality provided in the problem statement, we can see that for $ax^p$ dominates the function around the origin. Due to the absolute value in the inequality for $g(x)$, any value of $x$ put into $g(x)$, will return a positive number. When $p$ is odd, we can see that the exponent of $x$ is even. This means that even for negative values of $x$ the output will be positive. Therefore since both terms of the Lyapunov function derivative yeild positive numbers, $\dot{V}(x)$ can only be \textbf{negative semi definite} or better when the constant $a < 0$.

\subsection*{Case 2: ``P'' is odd \& ``a'' greater than 0 }

As with Case \#1, when the $p$ is odd. We can show that the sign of the Lyapunov derivative is negative \textbf{only} when the constant $a <0$. In this situation, where $a >0$, that means that $\dot{V}(x)$ is not negative semi definite or better and therefore must be unstable.

\subsection*{Case 3: ``P'' is even \& ``a'' is not equal to  0 }

In the even that $p$ is an even number, due to the formulation of $\dot{V}(x)$ we can see that the dominate part of the function $ax^{p+1}$ will be an exponent to an odd power because of the plus one. Since odd functions retain the sign of their argument, there will always be some region of this function (under these conditions) where the function is not negative semi definite or better therefore must be unstable. This result holds since even though the constant $a$ may change the sign of the function, the resultant output will carry the sign of its argument. Since the condition for stability does not hold all inputs in the statespace, it must be unstable. Naturally the case when $a=0$ is trivial since the resulting expression is valued at zero.   
