\section*{Problem 3}

Given the following systems, use a quadratic Lyapunov function candidate to show that the orgin isymptotically stable. Then investigate whether the origin is globally asymptotically stable

\subsection*{Part 1}

\begin{flalign*}
  & \dot{x}_1 = -x_1 + x_1x_2 \\
  & \dot{x}_2 = -x_2
\end{flalign*}

\subsubsection*{Part 1 - Asymptotic Stability}


\noindent Since the first section of this question is asking to verify that the equilibrium point is asymptotically stable, it will suffice to find the local domain $D$ where the conditions $V(0)=0$ and $\dot{V}(x) < 0$ $\forall x \in D -\{0\}$.\\

\noindent Assuming a quadratic function, we can compute the Lyapunov Function as...

\begin{flalign*}
& V(x)=\frac{1}{2} x_{1}^{2}+\frac{1}{2} x_{2}^{2} \\
& \dot{V}(x)=x_{1} \dot{x}_{1}+x_{2} \dot{x}_{2} \\
& \text{such that...}\\
& \\
& \dot{V}(x)=-x_{1}^{2}+x_{1}^{2} x_{2}-x_{2}^{2}\\
&
\end{flalign*}

\noindent To show that function, $V(x)$ is asymptotically stable, we can show that (1) $V(0) = 0$ and that (2) $\dot{V}(x) < 0, \: \: \forall x \in D - \{ 0 \},$ according to Lyapunov Theorem. It is trivial to show that the function $V(0) = 0$, and that $V(x)>0$, since the function candidate is given as quadratic. Therefore the problem is to demonstrate that the time derivative of $V$ is \textbf{negative definite}.\\

\noindent In order to establish asymptotical stability, we only need to demonstrate that the stability exists for some domain $D \subset \R^2$. To test the function in domain $D$ we can assume that there exists a ball such that... \\

$$
B_r = \{ x \in \R^2 | \left\Vert x \right\Vert < r\}
$$

\noindent If we can show that if there exists a radius of $r$ for which $\dot{V}(x)$ is negative definite, then have have effectively shown that the domain $D$ exists and proven asymptotic stability via Lyapunov Theorem.

\indent Given the condition...

$$
\dot{V}(x)=-x_{1}^{2}+x_{1}^{2} x_{2}-x_{2}^{2} < 0
$$

\noindent For the inequality to hold...

$$
x_{1}^{2} x_{2} < x_{1}^{2} + x_{2}^{2}
$$

\noindent By using the definition of the ball $B_r$, we can show that for a given radius $r$, $\left\Vert x_1 \right\Vert < r$ \& $\left\Vert x_2 \right\Vert < r$, for which we can rewrite the inequality as ...

$$
r^3 < r^2 + r^2
$$

\noindent After simplification we find that...

$$
r < 2
$$

\noindent Therefore for a given radius of $r < 2$ the system is \textbf{asymptotically stable}.


\subsubsection*{Part 1 - Global Stability}

\noindent In order to demonstrate Globally Asymptotic Stability, we can use the Lyapunov Theorem for Global Stability which can be stated as follows.

\begin{equation}
\begin{array}{c}
V(0)=0 \text { and } V(x)>0, \quad \forall x \neq 0 \\
\|x\| \rightarrow \infty \Rightarrow V(x) \rightarrow \infty \\
\dot{V}(x)<0, \quad \forall x \neq 0
\end{array}
\end{equation}

\noindent We have shown in the previous problem that there exist conditions where $\dot{V}$ is negative definite $\dot{V}(x) < -x_{1}^2 - x_{2}^2 + r|x_1||x_2| < 0$, when $r<2$ and by inspection we can see that the Lyapunov candiate function is `radially unbounded'. By demonstrating these condition, we can use the theorem and say that the function must be globally asymptotically stable.

\subsection*{Part 2}

\begin{flalign*}
  & \dot{x}_1 = -x_2 + x_1(1 - x_1^2-x_2^2) \\
  & \dot{x}_2 = x_1 - x_2(1 - x_1^2-x_2^2)
\end{flalign*}


\subsubsection*{Part 2 - Asymptotic Stability}

\noindent The Lyapunov function candidate and its time derivate can be given by...

\begin{equation}
\begin{array}{l}
V(x)=\frac{1}{2} x_{1}^{2}+\frac{1}{2} x_{2}^{2} \\
\dot{V}(x)=x_{1} \dot{x}_{1}+x_{2} \dot{x}_{2}=x_{1}\left(-x_{2}-x_{1}\left[1-x_{1}^{2}-x_{2}^{2}\right]\right)+\ldots \\
\ldots+x_{2}\left(x_{1}-x_{2}\left[1-x_{1}^{2}-x_{2}^{2}\right]\right)
\end{array}
\end{equation}

\noindent When simplifed this functions yield the expression....

$$\dot{V}(x) = \left( -x_1^2 - x_2^2 \right)\left[ 1 - x_1^2 - x_2^2\right]$$

\noindent The first expression in this term is always negative definite due to the coefficients in front of $x_1$ and $x_2$. This implies that in order for the function to be negative definite the term $ 1 - x_1^2 - x_2^2 > 0 $ since the multiplicitive negative would negate the previous term. This means that we need to show the second term is positive definite, for the complete inequality to hold.

\noindent By rearrangement...

$$
x_1^2 + x_2^2 < 1 \: \: \forall x \in \R^{+}
$$

\noindent The only way for this to be true is if the values of $x_1$ and $x_2$ are bounded from above by $\sqrt{.5}$. This can also be shown by using the definition of the function $V$ and substituting it into the expression for $\dot{V}(x)$. Which yields.


$$
\dot{V}(x) = -2\cdot{V}\left[ 1 - 2\cdot V \right] < 0
$$

\noindent By simplification the result must be that the value of $V < \frac{1}{2}$. This is equivalent to the bound placed on $x_1$ and $x_2$ above.

\subsubsection*{Part 2 - Global Stability}

\noindent By the very definition of the previous stability result, we showed that $V \ngeq \frac{1}{2}$. Since this is the requirement for obtaining the negative definite $\dot{V}$. Therefore the solution is \textbf{not} globally asymptotically stable, since it is not radially unbounded on $\R^2$.

\subsection*{Part 3}

\begin{flalign*}
  & \dot{x}_1 = x_2(1-x_2^2) \\
  & \dot{x}_2 = -(x_1+x_2)(1-x_2^2)
\end{flalign*}

\subsubsection*{Part 3 - Asymptotic Stability}

\noindent Given the quadratic Lyapunov function, and its corresponding time derivative, we obtain the following expression for the given system.

\begin{equation}
\begin{array}{l}
\dot{V}(x) =x_{1} \dot{x}_{1}+x_{2} \dot{x}_{2} \\
\quad \quad \:\: =x_{1}\left(x_{2}\left[1-x_{1}^{2}\right]\right)+x_{2}\left(-\left(x_{1}+x_{2}\right)\left[1-x_{1}^{2}\right]\right)
\end{array}
\end{equation}


\noindent This expression can be reduced to..

$$
\dot{V(x)}=-x_2^2\left[ 1 - x_1^2 \right]
$$

\noindent Since this first part of this expression is always negative definite $x\in \R^2$, we know that the second term must therefore be positive definite for the function to remain negative due to the multiplicitive negative. This requires...

$$
1 - x_1^2 > 0
$$

\noindent Which futher simplies to the fact that $x_1 < 1$ for the function to remain negative definite. Under this condition, the Lyapunov function candiate follows all of the conditions for given in Lyapunov's Stability Theorem and therefore is \textbf{asymptotically stable}.

\subsubsection*{Part 3 - Global Stability}

\noindent Since the time derivate of $V$ is not negative definite for $x \in \R^2$, due to the restriction that $x_1 <1$. The system is \textbf{not globally asymptotically stable}.


\subsection*{Part 4}

\begin{flalign*}
  & \dot{x}_1 = -x_1 -x_2 \\
  & \dot{x}_2 = 2x_1 - x_2^3
\end{flalign*}

\noindent By using the following Lyapunov Candidate function...

$$
V(x) = x_1^2 + \frac{1}{2}x_2^2
$$


\noindent We can show that its time derivative will be ...

$$
\begin{array}{l}
\dot{V}=2 x_{1} \dot{x}_{1}+x_{2} \dot{x}_{2} \\
\dot{V}=2 x_{1}\left(-x_{1}-x_{2}\right)+x_{2}\left(2 x_{1}-x_{2}^{3}\right) \\
\end{array}
$$

\noindent After simplication the Lyapunov Function is shown to be...

$$
\dot{V} = -2x_1^2 -x_1^4
$$

\noindent Since this function is \underline{negative definite} such that $\forall x \in D - \{0 \} | \dot{V} < 0$ we can conclude (via Lyapunov Stability Theorem) that the origin is \underline{\textbf{asymptotically stable}}  However, we can go beyond this conclusion, since this previous statement is valid for when $\forall x \in \R^2$, and since the quadratic candidate function is radially unbounded with respect to $x$, we can conlude (via Lyapunov Global Stability Theorem) that the system is \underline{\textbf{Globally Asymptotically Stable}} aout the origin. 
