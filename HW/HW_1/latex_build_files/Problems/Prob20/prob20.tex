\section*{Problem \#20}
A function of $f$ is \textbf{uniformly continuous} if $\forall \: \epsilon > 0 \: \text{ then } \exists \: \delta > 0 \text{ such that }$

$$ \left\Vert f(y) - f(x) \right\Vert < \epsilon \hspace{1cm} \text{ whenever} $$
$$ \left\Vert y - x \right\Vert < \delta $$

\noindent Furthermore, a function $f$ is \textbf{Lipschitz Continuous} if $\exists \: L < \infty$ such that...

$$
\left\Vert f(y) - f(x) \right\Vert \leq L \left\Vert y -x \right\Vert
$$

\noindent We can combine the two statements be defining...

$$ \delta = \frac{\epsilon}{L} $$

\noindent By rewriting the condition for Lipschitz Continuity, and including the expression above, we can show that the `uniform continuity' of a function is `essentially' a form of the Lipschitz condition.

$$ \left\Vert y - x \right\Vert < \delta \Rightarrow \left\Vert f(y) - f(x) \right\Vert \leq L \left\Vert y - x \right\Vert < \epsilon   $$


\noindent Therefore, while being \underline{uniformly continuous} does \textbf{not} imply that a function is Lipschitz Continuous, the fact that Lipschitz Continuous is a stronger assumption on the continuity of a function, any Lipschitz function implies that that function is uniformly continuous. 
