
\section*{Problem \#15}

\subsubsection*{15.1.A}
For ...

$$
f(x) =
\begin{cases}
    x^{2}sin(1/x), & \text{for $x\neq0 $} \\
    0, & \text{for $x =0$}
\end{cases}
$$

The function is \textbf{not} continuously differentiable since even though the derivative for the function exists there is an essential discontinuity at $x =0$ for the derivative of the function, where the limit does not exist. As such it is not Continuously Differentiable.
\subsubsection*{15.1.B}

The function $f(x)$ \textbf{is} locally Lipschitz at $x=0$, since there exists an `L' which satisfies the Lipschitz inequality, for every possible point $(x_1, x_2) \in B_r  $.

\subsubsection*{15.1.C}

The function $f(x)$ is not strictly speaking `continuous' at $x=0$, since it is a piecewise function, with an obvious discontinuity at $x=0$, however in so much as the domain of continuity is $x=0$, the \textbf{yes} its continuous at that point.

\subsubsection*{15.1.D}

The function $f(x)$ is \textbf{not} globally Lipschitz continuous since there is no `single' constant L which satisfies the Lipschitz inequality for each point $(x,y) \in \R$.

\subsubsection*{15.1.E}

The function $f(x)$ is \textbf{not} `uniformly continuous' on $\R$. It would be uniformly continuous on a bounded interval.

\subsubsection*{15.1.F}

The function $f(x)$ is Lipschitz on the interval $[-1, 1]$, since there exists a `L' over that interval which satisfies the Lipschitz inequality.


\subsection{}

For ...

$$
f(x) =
\begin{cases}
    x^{3}sin(1/x), & \text{for $x\neq0 $} \\
    0, & \text{for $x =0$}
\end{cases}
$$


\subsubsection*{15.2.A}

The function $f$ is differentiable over its entire domain, however, there is discontinuity at $x=0$. However, when the domain of the continuity is restricted to the point $x=0$, the function \textbf{is} continuously differentiable at that point.

\subsubsection*{15.2.B}

The function $f$ is Locally Lipschitz at the point $x=0$.

\subsubsection*{15.2.C}

The function $f$, is not continuous over its entire domain; however, when restricted to the point $x=0$ the function \textbf{is} continuous at that point.

\subsubsection*{15.2.D}
The functino $f$ is \textbf{not} gloablly Lipschitz since its derivative is not boundedin $\R$.
\subsubsection*{15.2.E}

The function $f$ is \textbf{not} uniformly continuous since the there exists a $\delta$ which does not satisfy the condition for uniform continuity given an $\epsilon$.
\subsubsection*{15.2.F}

The function $f$ \textbf{is} Lipschitz on the bounded interval $[-1 , 1]$ since there exists an L which can bound the function according to its definition.


\subsection{}

For ...

$$
f(x) = tan\left( \frac{\pi x}{2}\right)
$$

\subsubsection*{15.3.A}
The functino $f$ \textbf{is} continuously differentiable at $x=0$, since both the function and derivative of $f$ exist when taken in the limit at $x=0$.

\subsubsection*{15.3.B}

Given $y =.25$ and $x=0$

$$
\left\Vert f(y) - f(x) \right\Vert \leq L \left\Vert y -x \right\Vert
$$

$$
\left\Vert .25534 - 0 \right\Vert \leq L \left\Vert .25 - 0 \right\Vert
$$

Where if $L =2$, the Lipschitz condition \textbf{is} satisfied in ball centered around the point $x=0$.

\subsubsection*{15.3.C}

Using the Limit evaluation of a function, it can be shown the that the function $f$ approaches $0$ in the limit. Therefore the function $f$ is continuous at $x=0$.

\subsubsection*{15.3.D}

The function $f$ is \textbf{not} globally Lipschitz, since its derivative is not bounded.

\subsubsection*{15.3.E}

The function $f$ is not uniformly continuous since, due to nature of the tanget function, the verticle asymptotes prevent there existing a single $\delta$ that satisfies the condition for uniform continuity given a single $\epsilon$.


\subsubsection*{15.3.F}
The functino $f$ \textbf{is} Lipschitz on the interval $[-1, 1]$.
