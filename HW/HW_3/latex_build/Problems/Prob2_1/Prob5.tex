\section*{Problem \#2.1}


The first task is to check whether the \textbf{\underline{unforced}} system is stable.

$$
\begin{array}{l}
\dot{x}_{1}=-x_{1} + x_{1}^{2} x_{2} \\
\dot{x}_{2}=-x_{1}^{3}-x_{2}+u
\end{array}
$$

\noindent Where $V(x) = .5 \left\Vert x \right\Vert^2_2$, such that $\dot{V} = x_1\dot{x_1} + x_2 \dot{x_2}$. By plugging in the state equation we get...

$$
\dot{V}(x) = -x_1(-x_1 + x_1^2x_2) + x_2 (-x_1^3 -x_2 + u)
$$

$$
\dot{V}(x) = -x_1^2 -x_2^2 + x_2u
$$

\noindent For the unforced system wher $u \equiv 0$, it is obvious that the candidate function is positive definite and radially unbounded while the Lyapunov function is negative definite. Therefore the origin of the \underline{unforced} system is \textbf{Globally Asymptotically Stable}.

\noindent When the input is not identically zero, we can show that ...

$$
\dot{V}(x) =  -\left\Vert x \right \Vert_2^2 + \left\Vert x_2\right\Vert|u|
$$

\noindent We can show  with the following parameterization that that this system is stable under the condition the following condition


$$
-(1-\theta)\left\Vert x \right \Vert_2^2 -\theta\left\Vert x \right \Vert_2^2 + \left\Vert x_2\right\Vert|u|
$$

\noindent We can show that this function can be upper bounded by ...

$$
-(1-\theta)\left\Vert x \right \Vert_2^2 -\theta\left\Vert x \right \Vert_2^2 + \left\Vert x \right\Vert_2|u|
$$

\noindent We can now detemine the condition such that the positive term in the previous function will always remain less than or equal to zero.

$$
-(1-\theta)\left\Vert x \right \Vert_2^2 -\theta\left\Vert x \right \Vert_2^2 + \left\Vert x \right\Vert_2|u| \leq -(1-\theta)\left\Vert x \right \Vert_2^2
$$


\noindent After simplication we can conclude that...


$$
\forall \left\Vert x\right\Vert \geq \frac{|u|}{\theta}
$$

\noindent From which we can infer that the system is \underline{\textbf{locally ISS}}; however in order to determine the nature of globally ISS, we need to show that, $f \in C_1$, globally Lipschitz in $(x,u)$, uniformly in $t,x =0$ is G.E.S. for $u \equiv 0$  $\rightarrow \dot{x} = f(t,x,u)$


\noindent Given this definition we can show using Theorem 4.10 that the system is actually G.E.S. when the input $u \equiv 0$, since we can show that we can bound $V(x)$ using $K_1$ and $K_2$, and since we can further show that we can bound $\dot{V}(x)$ with a constant $K_3$ with a given constant $a = 2$. Therefore we can conlcude that the system is \underline{\textbf{globally ISS}}.
