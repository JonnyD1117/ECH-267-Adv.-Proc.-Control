\section*{Problem 1.3}


The first task is to check whether the \textbf{\underline{unforced}} system is stable.

$$
\dot{x} = -x
$$

\noindent Where $V(x) = .5x^2$, such that $\dot{V} = x\dot{x}$. By plugging in the state equation we get...

$$
\dot{V}(x) = -x^2
$$

\noindent Obviously, the candidate function is positive definite and radially unbounded while the Lyapunov function is negative definite. Therefore the \underline{unforced} system is \textbf{Globally Asymptotically Stable}.

\noindent Now we can look at the forced behavior of the system given by


$$
\dot{x} = -x + x^2u
$$

\noindent Where $V(x) = .5x^2$, such that $\dot{V} = x\dot{x}$. By plugging in the state equation we get...

$$
\dot{V}(x) = x(-x + x^2u)
$$

\noindent This expands to ..

$$
\dot{V}(x) = -x^2 + x^3u
$$

\noindent To determine the bound such that the input to the system never causes the system to become unstable we use $\theta \in (0,1)$ to parameterize the system as follows.


$$
\dot{V}(x) = -(1-\theta)x^2 - \theta x^2 + x^3u
$$

\noindent Since we want this function to be negative definite, we want to ensure that $ \dot{V}(x) \leq -(1-\theta)x^2 $. Which enables us to write the expression as follows...


$$
\dot{V}(x) = -(1-\theta)x^2 - \theta x^2 + x^3u \leq -(1-\theta)x^2
$$

\noindent This expression simplifies to...

$$
\dot{V}(x) = - \theta x^2 + x^3u \leq 0
$$

\noindent By using this expression we can detemine the bounds for which the inputs of the system will keep the system asymptotically stable.

$$
-\theta + ux \leq 0
$$

$$
ux \leq 0
$$


$$
|x| \leq \frac{\theta}{|u|}
$$

\noindent Under this condition the system is globally asymptotically stable, and therefore we can show via Theorem 4.19 that the system is \underline{\textbf{ISS}}.
