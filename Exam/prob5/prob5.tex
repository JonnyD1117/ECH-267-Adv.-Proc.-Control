\section*{Problem \#5}

Show that the origin of the system

$$
\begin{array}{l}
\dot{x}_{1}=x_{2} \\
\dot{x}_{2}=-x_{1}^{3}-x_{2}^{3}
\end{array}
$$

\noindent is globally asymptotically stable, using a suitable Lyapunov function.


\subsection*{Solution Problem 5}

In order to determine the stability of the origin of this system, we must first develope a Lyapunov Candidate function $V(x)$ that is postive definite.

$$
V(x) = \frac{1}{4}x_1^4 + \frac{1}{2}x_2^2
$$

\noindent Which results in the following Lyapunov function...

$$
\dot{V}(x) = x_1^3\dot{x}_1 + x_2\dot{x}_2
$$

\noindent By substituing the state equations back into $\dot{V}$, we obtain...

$$
\begin{aligned}
  \dot{V}(x) & = x_1^3(x_2) + x_2 (-x_1^3 -x_2^3) \\
        & = -x_2^4
\end{aligned}
$$

\noindent Unfortunately, we can only state that this function is negative definite since, given $x_2 =0$ we can show that $\forall x_1 \in \R$, the Lyapunov function is equal to zero. Since this function doesn't have the property to that $\dot{V}< 0$ and $V(0)=0$ at the origin $x = (0,0)$, we must conclude that the function is only negative semi-definite. \\

\noindent This is unfortunate since we cannot use Lyapunov's Global Stability Theorem inorder to demonstrate that the system is globally asymptotically stable. In stead we must use \textbf{4.4 La Salle's Theorem} and its following \textbf{Corolarry 4.2} (Khalil), to demonstrate any further properties about the system. \\

From the previous analysis we have shown the following properties

\begin{itemize}
  \item $V: \R^2 \rightarrow \R$ (Defined on a global domain)
  \item $V \in C_1$ (Continuously differentiable)
  \item $V(x)\rightarrow \infty$ as $\left\Vert x \right\Vert \rightarrow \infty$ (Radially unbounded)
  \item $V(x) > 0$ and $V(0) = 0$ (Positive Definite)
\end{itemize}

\noindent The only limitation we have is that $\dot{V}(x)$ is only negative semi-definite and NOT negative definite. By applying La Salle's Theorem (and its Corolarry) we can show that...

$$
S = \{ x\in \R^2 | \dot{V}(x) = 0 \}
$$

\noindent By solving the equation $\dot{V}(x) = -x_2^4 = 0$, we can show that $x_2 \equiv 0 $ and therefore...

$$
S = \{ x_2 = 0 \}
$$

\noindent However, we can show that by plugging this result into the state equations that

$$
x_2 \equiv 0 \Rightarrow x_1 = 0
$$

\noindent Since there is no solution that can stay identically in $S$ other than the origin $x(t) = (0,0)$, then by \textbf{Corolarry 4.2} we can conclude that the system is \underline{\textbf{Globally Asymptotically Stable}}.
