\documentclass[journal]{IEEEtran}

\usepackage{biblatex}
\addbibresource{citations.bib}

% \hyphenation{op-tical net-works semi-conduc-tor}


\begin{document}

\title{Nonlinear MPC for Robotic Arm Path Planning \\ ECH-267 Final Project Report}


\author{Jonathan~Dorsey}



% The paper headers
\markboth{Journal of Graduate Schooling Assignments, March~2021}%
{Dorsey \MakeLowercase{\textit{et al.}}: Nonlinear MPC for Robotic Arm Path Planning ECH-267 Final Project Report}

% make the title area
\maketitle

% As a general rule, do not put math, special symbols or citations
% in the abstract or keywords.
\begin{abstract}
  The objective of this project is to implement simulated optimal \underline{Path Planner} using Model Predictive Control (MPC) to plan and control the behavior of a 3 degree of freedom (3DOF) robotic arm. The responsibility of the MPC planner will be to generate the `optimal' path and to drive the arm from its current position to the next. The main challenges faced in completing this project consist of solving for the \textbf{nonlinear equations of motion} (as well as any required forward/inverse kinematics) of the robotic arm as well as formulating and solving the \textbf{MPC controller}, at each timestep, both control and planning using the CasADi optimization framework, to test the scenario of simulated real-time performance.
\end{abstract}

% Note that keywords are not normally used for peerreview papers.
\begin{IEEEkeywords}
Model Predictive Control, Robot Arm, Lagrange Equations, Path Planning, Obsticle Avoidance, .
\end{IEEEkeywords}


\IEEEpeerreviewmaketitle

\section{Introduction}

\IEEEPARstart{T}{his} objective of this paper is to design, program, and test the use of MPC for near real-time path planning and control, for robotic arm. In the world of robotics, many constraints are placed upon a robotic system, such as metrics of performance, physical construction and limitation of the hardware, it performance, and soft constraints such as safety around unexpeted obsticles such as humans, other robots, or miscellaneous obsticles. While the subjects of model predictive control (MPC) and robotic manipulators (for both design and control), there is exists a lack of targeted literature covering the question about how MPC and robotics can be used in the domain of real-time path planning. It has long be known that MPC is viable controller with many flavors and versions for different applications; however, in practice MPC only sees limited use due to the computational lag resulting from using numerical optimization as the basis for feedback control. To this end, MPC might be too slow for use in the control of a system, but if the optimal trajectory can be estimated using MPC, other controllers might be better suited for the task of tracking the optimal path which the MPC controller is able to generate. Since this path is only updated as needed, the speed of the MPC solution is no longer a significant obsticle since we can effectively offload the task of controller to a less computationally expensive methodology.


\subsection{Subsection Heading Here}
Subsection text here.

\subsubsection{Subsubsection Heading Here}
Subsubsection text here.


\section{Background}

Some background information
\section{Literature Review}
Literature review goes here

\subsection{Model Predictive Control}

\subsection{Path Planning}

\subsection{Robotics}

\subsubsection{Equations of Motion}

\subsubsection{Denavit-Hartenberg Paramters}

\subsubsection{Forward Kinematics}



\section{MPC Formulation}

Describe MPC formulation

\section{Path Planner}

How to use MPC as a Path Planner

\section{Dynamic Model}

Derive or explain the dynamics model

\subsection{Nonlinear Model}

\subsection{Linearized Model}

\section{Tracking Controllers}

 \subsection{PID Control}

 \subsection{LQR}

 \subsection{MPC}

 \subsection{H \infity}

 \subsection{Feedback Linerization}

 \section{Results}

Results go here

\section{Discussion}

Dicussion and points of note.


\section{Conclusion}
The conclusion goes here.


\appendices
\section{Proof of the First Zonklar Equation}
Appendix one text goes here.


\section{}
Appendix two text goes here.


\section*{Acknowledgment}


The authors would like to thank Mountain Dew and his matress for constant support and comfort.



\ifCLASSOPTIONcaptionsoff
  \newpage
\fi



% \begin{thebibliography}{1}
%
% \bibitem{IEEEhowto:kopka}
% H.~Kopka and P.~W. Daly, \emph{A Guide to \LaTeX}, 3rd~ed.\hskip 1em plus
%   0.5em minus 0.4em\relax Harlow, England: Addison-Wesley, 1999.
%
%       \bibitem{}  J. B. Rawlings, M. M. Diehl, and D. Q. Mayne, Model predictive control: theory, computation and design. Madison: Nob Hill, 2017.
%
%       \bibitem{} B. Armstrong, O. Khatib and J. Burdick, "The explicit dynamic model and inertial parameters of the PUMA 560 arm," Proceedings. 1986 IEEE International Conference on Robotics and Automation, San Francisco, CA, USA, 1986, pp. 510-518, doi: 10.1109/ROBOT.1986.1087644.
%
% \end{thebibliography}
\cite{craig_introduction_2005}
\cite{khalil_nonlinear_2002}
\cite{rawlings_model_2017}
\cite{armstrong_explicit_1986}
\cite{ogata_modern_2010}
\cite{meriam_engineering_1993}
\cite{greenwood_advanced_2006}
\cite{borrelli_predictive_2017}
\cite{boyd_convex_2004}
\cite{slotine_applied_1991}

\printbibliography

\begin{IEEEbiographynophoto}{Jonathan Dorsey}
Biography text here.
\end{IEEEbiographynophoto}

\end{document}
